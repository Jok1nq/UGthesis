\cleardoublepage{}
\begin{center}
    \bfseries \zihao{3} 致~谢
\end{center}

时光飞逝,本科四年生活转眼已接近尾声。
四年间,在数学分析、操作系统等课程的浸润下,
在实验室科研的洗礼下,我走出了自己的舒适圈,为自己打开了一个新的世界。
在本科期间,我渐渐完善了自己的三观,沉淀了自己的想法,对人生和责任有着更多的思考与理解。
即将毕业,我想我有勇气对自己说,我已经准备好开始下一段旅程。

落笔生花,易画时间百态;收笔成憾,难写一生师恩。

首先,特别感谢我的导师纪守领老师。纪守领老师不仅是我竺院的荣誉导师,毕设指导教师,更是我人生路上的宝贵导师。
大二时,我懵懵懂懂做了浅显的学习与调研,选择了信息安全这一方向进行本科科研。
很荣幸纪老师能担任我的竺院荣誉导师,指导我的科研工作。
从此,我正式进入了实验室,开始参与科研工作。
在最初的科研生活中,我对自己的能力充满了怀疑,一切似乎都充满了挑战,都在自己的舒适区之外。
纪老师耐心鼓励我,让我认识到了研究需要静下心来,认真思考,多一点韧性,少一点任性。
有时偷懒时,纪老师会教育我要提高自己的执行力;
在遇到困难时,纪老师会提醒我不要畏难,很多事都没有想象的那么难,着手去做就好;
科研不顺时,纪老师会用自己经历鼓励我,厚积薄发,有沉淀才能有突破。
在决定赴海外攻读博士学位后,纪老师也给了我莫大的帮助,从写推荐信到简历的修订。

桃李春风一杯酒,江湖夜雨十年灯。

此外,感谢在NESA Lab带领我进入科研大门,并全程指导我完成毕业论文的常博宇学长。
他认真勤奋、一丝不苟、头脑清晰,以高标准严格要求自己,他时刻以实际行动感染着我。
在他的带领下,我掌握了基础的科研能力,学到了科研必要的精神,锻炼了自己处理压力的能力,提高了自己的韧性。
同时也感谢梁红、赵彬彬、付冲、张凌铭、刘丁豪、何平、许嘉诚等众多NESA Lab的学长学姐对本文工作的建议和意见。
很荣幸能在本科期间加入NESA Lab实验室,一次次学习与交流中,我不仅在科研学习上有了质的提升,也在性格和人生观上有了很多好的改变。

慈母手中线,游子身上衣。

另外,要感谢我的父母,在我整个成长和学习过程中,父母总在背后坚定地支持我。
也有责备,也有鼓励,在过去的二十年中,我始终是父母的孩子,似乎从未长大。
本科期间,我开始接触到小学中学期间从未接触过的各种困难与挫折,害怕过,也动摇过。
每每此时,父母都会坚定站出来,鼓励我的意志,帮我承担困难,帮助我成长。
如今的成就每一步,都离不开父母的陪伴与支持。

人生如逆旅,我亦是行人。

最后,我要感谢我自己。
回首四年时光,有欢乐,有荣耀,也有伤感,也有困苦。
感谢自己在挫折的洗礼下坚持下来,坚信自己的优秀,坚信自己的能力,
感谢自己在挫折中培养了自己的韧性与处理压力的能力,
感谢时刻陪伴自己的自己。

回首向来萧瑟处,前行,无畏风雨心有晴。

